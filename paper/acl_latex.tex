\documentclass[preprint]{article}

% Change "review" to "final" to generate the final (sometimes called camera-ready) version.
% Change to "preprint" to generate a non-anonymous version with page numbers.
\usepackage[review]{acl}

% Standard package includes
\usepackage{times}
\usepackage{latexsym}

% For proper rendering and hyphenation of words containing Latin characters (including in bib files)
\usepackage[T1]{fontenc}
% For Vietnamese characters
% \usepackage[T5]{fontenc}
% See https://www.latex-project.org/help/documentation/encguide.pdf for other character sets

% This assumes your files are encoded as UTF8
\usepackage[utf8]{inputenc}

% This is not strictly necessary, and may be commented out,
% but it will improve the layout of the manuscript,
% and will typically save some space.
\usepackage{microtype}

% This is also not strictly necessary, and may be commented out.
% However, it will improve the aesthetics of text in
% the typewriter font.
\usepackage{inconsolata}

%Including images in your LaTeX document requires adding
%additional package(s)
\usepackage{graphicx}

% If the title and author information does not fit in the area allocated, uncomment the following
%
%\setlength\titlebox{<dim>}
%
% and set <dim> to something 5cm or larger.

\title{Analysis of Linguistic Stereotypes in GenerativeAI}

% Author information can be set in various styles:
% For several authors from the same institution:
% \author{Author 1 \and ... \and Author n \\
%         Address line \\ ... \\ Address line}
% if the names do not fit well on one line use
%         Author 1 \\ {\bf Author 2} \\ ... \\ {\bf Author n} \\
% For authors from different institutions:
% \author{Author 1 \\ Address line \\  ... \\ Address line
%         \And  ... \And
%         Author n \\ Address line \\ ... \\ Address line}
% To start a separate ``row'' of authors use \AND, as in
% \author{Author 1 \\ Address line \\  ... \\ Address line
%         \AND
%         Author 2 \\ Address line \\ ... \\ Address line \And
%         Author 3 \\ Address line \\ ... \\ Address line}

%\author{Adriano De Cesare \\
  %\texttt{s333044@studenti.polito.it} \\\And
  %Anna Lisa Maddaloni \\
  %\texttt{s333037@studenti.polito} \\\And
  %Giovanni Giordano \\
  %\texttt{s331574@studenti.polito.it} \\\And
  %Matteo Di Gregorio \\
  %\texttt{s333943@studenti.polito.it} \\\And
  %Silvia Mantione} \\
  %\texttt{s333955@studenti.polito.it \\}}

  \author{Adriano De Cesare \\\And
  %\texttt{s333044@studenti.polito.it} \\\And
  Anna Lisa Maddaloni \\\And
  %\texttt{s333037@studenti.polito} \\\And
  Giovanni Giordano \\hh
  %\texttt{s331574@studenti.polito.it} \\\And
  Matteo Di Gregorio \\\And
  %\texttt{s333943@studenti.polito.it} \\\And
  Silvia Mantione\\} 
  %\texttt{s333955@studenti.polito.it \\}}


%\author{
%  \textbf{First Author\textsuperscript{1}},
%  \textbf{Second Author\textsuperscript{1,2}},
%  \textbf{Third T. Author\textsuperscript{1}},
%  \textbf{Fourth Author\textsuperscript{1}},
%\\
%  \textbf{Fifth Author\textsuperscript{1,2}},
%  \textbf{Sixth Author\textsuperscript{1}},
%  \textbf{Seventh Author\textsuperscript{1}},
%  \textbf{Eighth Author \textsuperscript{1,2,3,4}},
%\\
%  \textbf{Ninth Author\textsuperscript{1}},
%  \textbf{Tenth Author\textsuperscript{1}},
%  \textbf{Eleventh E. Author\textsuperscript{1,2,3,4,5}},
%  \textbf{Twelfth Author\textsuperscript{1}},
%\\
%  \textbf{Thirteenth Author\textsuperscript{3}},
%  \textbf{Fourteenth F. Author\textsuperscript{2,4}},
%  \textbf{Fifteenth Author\textsuperscript{1}},
%  \textbf{Sixteenth Author\textsuperscript{1}},
%\\
%  \textbf{Seventeenth S. Author\textsuperscript{4,5}},
%  \textbf{Eighteenth Author\textsuperscript{3,4}},
%  \textbf{Nineteenth N. Author\textsuperscript{2,5}},
%  \textbf{Twentieth Author\textsuperscript{1}}
%\\
%\\
%  \textsuperscript{1}Affiliation 1,
%  \textsuperscript{2}Affiliation 2,
%  \textsuperscript{3}Affiliation 3,
%  \textsuperscript{4}Affiliation 4,
%  \textsuperscript{5}Affiliation 5
%\\
%  \small{
%    \textbf{Correspondence:} \href{mailto:email@domain}{email@domain}
%  }
%}

\begin{document}
\maketitle
\begin{abstract}
This document is a supplement to the general instructions for *ACL authors. It contains instructions for using the \LaTeX{} style files for ACL conferences.
The document itself conforms to its own specifications, and is therefore an example of what your manuscript should look like.
These instructions should be used both for papers submitted for review and for final versions of accepted papers.
\end{abstract}

\section{Introduction}

Describe here the objectives and the context of application of your experiment. This text can be based on the application description or on the SEMeval task requirements

\subsection{Research Questions}

\textbf{R.Q.1: }What types of linguistic stereotypes do LLMs reproduce?
\\ \textbf{R.Q.2: }Does prompt structure (zero-shot, role prompting, chain-of-thought) amplify or reduce bias?
\\ \textbf{R.Q.3: }Can multi-agent critique frameworks reduce stereotypical outputs?


\section{Background}
The ability of Large Language Models (LLMs) to pick up biases encoded in training data can lead to the risk of systematic judgments on social stereotypes,reflecting different categories of biases, that can be based on gender or on race. (\citet{bender-2021})  
\\ In the context of linguistic stereotypes, more recent students have highlighted covert forms of prejudice emerging through language usage itself, rather than through explicit mentioning. More into details, the authors have focused on African American English (AAE) and Standard American English(SAE). The study demonstrates that dialect prejudice has visible consequences: LLMs are more likely to assign AAE speakers to lower-prestige jobs, predict criminal behavior, or recommend harsher legal outcomes. 
(\citet{hofmann-24})  
\\Beyond English, studies on German dialects show that LLMs match dialect speakers with negative traits and occupational assignment with a lower score.
(\citet{bui-25}) 
\\Analogous patterns emerge in work on Egyptian Arabic, where LLMs exhibit higher bias compared to Modern Standard Arabic, reflecting biases against low-resource dialects. (\citet{elsafoury-25}) 
\\However, existing research conducted on Italian linguistic variations, focuses primarily on the model’s metalinguistic awareness and understanding of non-standard linguistic structures. (\citet{massaro-23}) 
\\ Building on the matched-guise methodology(\citet{hofmann-24}), this paper investigates covert stereotypes arising from the usage of Italian dialects.  



\section{System overview}
Describe all the methodological choices that you performed: e.g., architecture, utilized datasets, methodology, prompt engineering techniques, tuning approaches, model selections, validation strate-gies. Use separate subparagraphs for each piece of information. Use diagrams to describe architectural choices.


\section{Experimental results}
Describe all the results of your experimentation.
Provide separate subparagraphs for each of the defined research questions.

\section{Conclusions}
Provide a critical evaluation of your work. What are the main outcomes and limitations? How can the work be extended?



\subsection{Citations}


Table~\ref{citation-guide} shows the syntax supported by the style files.
We encourage you to use the natbib styles.
You can use the command \verb|\citet| (cite in text) to get ``author (year)'' citations, like this citation to a paper by \citet{Gusfield:97}.
You can use the command \verb|\citep| (cite in parentheses) to get ``(author, year)'' citations \citep{Gusfield:97}.
You can use the command \verb|\citealp| (alternative cite without parentheses) to get ``author, year'' citations, which is useful for using citations within parentheses (e.g. \citealp{Gusfield:97}).

A possessive citation can be made with the command \verb|\citeposs|.
This is not a standard natbib command, so it is generally not compatible
with other style files.



% Bibliography entries for the entire Anthology, followed by custom entries
%\bibliography{custom,anthology-overleaf-1,anthology-overleaf-2}

% Custom bibliography entries only
\bibliography{custom}


\end{document}
\bibliographystyle{acl_natbib}